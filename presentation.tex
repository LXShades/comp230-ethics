\documentclass{beamer}
\usetheme{Boadilla}

% Look at other themes?

\title{What are the most effective strategies to improve gender balance in game studios?}
\subtitle{Spoiler alert: This undergraduate student isn't going to solve this worldwide issue}
\author{Louis Foy}
\institute{Falmouth University}
\date{\today}

\begin{document}
\begin{frame}
	\titlepage
\end{frame}

\begin{frame}
	\frametitle{Question}
	\tableofcontents
\end{frame}

\section{Background}
\begin{frame}
	\frametitle{Background}
	\begin{itemize}
	    \item According to IGDA, women comprised 21\% of the game industry workforce in 2018 \cite{igda_satisfaction_2017} - down 1\% from 22\% in 2014 \cite{igda_satisfaction_2014}.
	    % (https://cdn.ymaws.com/www.igda.org/resource/collection/9215B88F-2AA3-4471-B44D-B5D58FF25DC7/IGDA_DSS_2014-Summary_Report.pdf)
	    % (https://cdn.ymaws.com/www.igda.org/resource/resmgr/2017_DSS_/!IGDA_DSS_2017_SummaryReport.pdf)
	    \item Similarly, in 2018, women comprised only 17\% of IT sectors in the UK \cite{uk_employees} % (https://www.ons.gov.uk/employmentandlabourmarket/peopleinwork/employmentandemployeetypes/datasets/employmentbyoccupationemp04)
	    \begin{itemize}
	        \item Management positions have the most balance: in 2018, the proportion of women in IT project and programme managers was 32\% \cite{uk_employees_2018}, up from 15\% in 2013 \cite{uk_employees_2013}.
	    \end{itemize}
	    \item Finally, according to HESA, only 17\% of computer science university students were female in 2016 \cite{hesa_2016} - corresponding with the IT industry
	    \begin{itemize}
	        \item This limitation in developing talent suggests the issue begins at a much earlier stage than the workplace.
	    \end{itemize}
	\end{itemize}
\end{frame}

\section{Issues}
\begin{frame}
    \frametitle{Issues}
    \begin{itemize}
        \item ``Sense of belonging'' is said to enhance meaning in life \cite{lambert_belonging_2013}. Much of what is described as `gamer culture', through regular incidents of specific harassment of women by vocal portions of the gaming community (such as GamerGate) \cite{riot_lawsuit, gonzalez_boysclub_2014}, is known to have a detrimental impact on the appeal of the game industry for women \cite{gonzalez_boysclub_2014}. (NEEDS MORE CITATION) %http://citeseerx.ist.psu.edu/viewdoc/download?doi=10.1.1.715.6087&rep=rep1&type=pdf
        %https://www.ideals.illinois.edu/bitstream/handle/2142/47355/325_ready.pdf?sequence=2
        \begin{itemize}
            \item However, the raised awareness of these issues combined with a dropping proportion of women in the industry, inspires the notion that discussion of these topics is not enough.
            \item Question: Could the greater awareness of workplace harrassment be reducing the appeal of the game industry to women through fear?
            \item Question: Could this discussion, by separating women and men from the social group, further polarise and threaten this sense of belonging?
        \end{itemize}
    \end{itemize}

\end{frame}

\section{Existing solutions}
\begin{frame}{Frame Title}
    % Find strengths and limitations of each existing solution
    % Can I propose a potential solution acknowledging strengths 
\end{frame}

\section{Outstanding Questions}
\begin{frame}
    \frametitle{Outstanding Questions}
    \item What are the links between game content and developer demographics?
    \begin{itemize}
        \item Hard to answer definitively: Demographics for players of specific games are hard to find.
        \item Anecdotal intuition may suggest a natural link between the gender of an artist and its fanbase, owing to differences in preference and perspective.
    \end{itemize}
    \item What is the association between the IT industry and gaming industry?
    \begin{itemize}
        \item A 2009 study \cite{betsy_questioning_2009} found that the link between gaming and working in the IT industry was quite significant, even though the gender disparity between both industries is similar.
        % https://dl.acm.org/citation.cfm?id=1536513.1536561
    \end{itemize}
\end{frame}

\nocite{*}
%\bibliographystyle{plain}
%\bibliography{references_louis}

\end{document}