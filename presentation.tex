\documentclass{beamer}
\usetheme{Rochester}

% Look at other themes?

\title{What are the most effective strategies to improve gender balance in game studios?}
\subtitle{A worldwide issue about to be completely and instantly solved by some random undergraduate student dancing on the heavy side of the gender balance scales}
\author{Louis Foy}
\institute{Falmouth University}
\date{\today}

\begin{document}
\begin{frame}
	\titlepage
\end{frame}

\begin{frame}
	\frametitle{Question}
	\tableofcontents
\end{frame}

\section{Background}
\begin{frame}
	\frametitle{Background}
	\begin{itemize}
	    \item According to IGDA, women comprised 21\% of the game industry workforce in 2018 \cite{igda_satisfaction_2017} - down 1\% from 22\% in 2014 \cite{igda_satisfaction_2014}.
	    \item Similarly, in 2018, women comprised only 17\% of IT sectors in the UK \cite{uk_employees_2018} % (https://www.ons.gov.uk/employmentandlabourmarket/peopleinwork/employmentandemployeetypes/datasets/employmentbyoccupationemp04)
	    \begin{itemize}
	        \item Management positions have the most balance: in 2018, the proportion of women in IT project and programme management positions was 32\% \cite{uk_employees_2018} (up from 15\% in 2013 \cite{uk_employees_2013}.)
	    \end{itemize}
	    \item Finally, according to HESA, only 17\% of computer science university students were female in 2016 \cite{hesa_2016} - corresponding with the IT industry
	    \begin{itemize}
	        \item This limitation in developing talent suggests the issue begins at a much earlier stage than the workplace.
	    \end{itemize}
	\end{itemize}
\end{frame}

\section{Issues}
\begin{frame}
    \frametitle{Issues}
    \begin{itemize}
        \item A 2013 study concluded that a ``sense of belonging'' is said to enhance meaning in life \cite{lambert_belonging_2013}
        \item Regular incidents of targeted harassment of women by vocal portions of the gaming community \cite{riot_lawsuit, gonzalez_boysclub_2014, perreault_gamergate_2018} threaten this sense of belonging.
        \begin{itemize}
        	\item Further research: GamerGate
        \end{itemize}
        \item In such an example of repeated harassment and the resulting controversy, a minority may be inclined to believe that this is a result of being outnumbered by an oppressive entity
        \begin{itemize}
        	\item This is unlikely to incentivise working in the industry as a minority
        \end{itemize}
    \end{itemize}
\end{frame}

\begin{frame}
	\frametitle{Issues - Part 2}
	\begin{itemize}
		\item Many studies and news articles push for greater character diversity in games.
		\item Is the appeal of playing games associated with the appeal of producing games?
		\begin{itemize}
			\item The rise in female gamers \cite{google_changethegame_2017} and drop in female developers \cite{igda_satisfaction_2014, igda_satisfaction_2017} challenges this association
		\end{itemize}
		\begin{itemize}
			\item Games that represent women more could improve the sense of belonging in the gaming community through role models - predicting the desire to make games. (cite 'playing games' linked to making games)
		\end{itemize}
	\end{itemize}
\end{frame}

\begin{frame}
	\frametitle{Effects of improved diversity}
	\begin{itemize}
		\item In an ideal world: could improve diversity of games due to increased diversity of perspectives in workplaces
		\begin{itemize}
			\item In the present world: actually can disrupt the implementation process due to creative differences \cite{harvey_diversity_2013}
			\item ``Transformational leadership'' can overcome this through moderation of ideas whilst facilitating intrinsic motivation within the team \cite{wang_transformational_2016}
		\end{itemize}
		\item An influential study analysing relational demography \cite{being_different} showed that workers in a heterogeneous workplace reported highest satisfaction when the organisation comprised members of their sex, as well as a general preference of homogeneity.
		\item A difference in gender, among other demographic differences, have been associated with a negative perception of subordinates by superiors overall \cite{subordinates}
		\item Conclusion: Existing behaviours disrupt potential benefits of diversity, at least in the present culture
	\end{itemize}
\end{frame}

\begin{frame}
\frametitle{Impacts of discussion}
\framesubtitle{``even though our society has become sensitized to negative sex stereotypes [...], it remains blind to its gratuitous emphasis on the gender dichotomy itself.'' - S. L. Bem \cite{gender_schema}}
\begin{itemize}
	\item Discussion of discrimination appears to be far from sufficient, but is necessary to facilitate the development of solutions
	\item Movements such as \#MeToo and \#GamerGate raise awareness of the harassment of women in the workplace
	\item Counter-productive effects:
	\begin{itemize}
		\item Could greater awareness of workplace harassment, while essential for progression, also successfully portray the workplace as an unappealing and unreachable place for women?
		\item Could sensitivity to the topic from conflict-fearing individuals, for fear of tension and self-association with a polarising or aggressive party, hinder progress?
	\end{itemize}
	%https://www.ahcafr.com/wp-content/uploads/2015/07/gender_schema_theory.pdf
\end{itemize}
\end{frame}

%\begin{frame}
%	\frametitle{Role models}
%	\begin{itemize}
%		\item Role models are believed to have a high impact on career choices (cite)
%		\begin{itemize}
%			\item A 2013 study 
			
%		\end{itemize}
% 		https://ac.els-cdn.com/S1877042813032825/1-s2.0-S1877042813032825-main.pdf?_tid=8b578109-533d-45ff-8393-8d31fd090b2f&acdnat=1541602167_5e684138c80f5219ddb572cebd27ca8b
%	\end{itemize}
%\end{frame}

\section{Potential solutions}
\begin{frame}{Potential solutions}
\begin{itemize}
	\item I had no time to cite these solution ideas, halp.
	\item Stats suggest the problem develops before university age.
	\item Several programmes (citation needed, HALP) have been engaged specifically to affirm and facilitate women's potential in computing, with K-12 level programming activities
	\begin{itemize}
		\item This helps the computing domain...
		\item ...however, demographics on sites such as DeviantART suggests a lot of talent exists already - but is not in the industry. This is indicative of the other issues at play.
	\end{itemize}
	\item Discussions portray a negative and uninviting perspectives of the industry
	\begin{itemize}
		\item How can we counter this with discussion of positive results?
	\end{itemize}
	\item Stronger harassment policies could be implemented to counter minority-targeted toxicity (and toxicity in general)
	\item Boom this problem is totally and cleanly resolved by this undergraduate student with Internets
    % Find strengths and limitations of each existing solution
    % Can I propose a potential solution acknowledging strengths 
\end{itemize}
\end{frame}

\section{Outstanding Questions}
\begin{frame}
    \frametitle{Outstanding Questions}
    \begin{itemize}
	    \item What are the links between game content and developer demographics?
	    \begin{itemize}
	        \item Hard to answer definitively: Demographics for players of specific games are hard to find.
	        \item Anecdotal intuition may suggest a natural link between the gender of an artist and its fanbase, owing to differences in preference and perspective.
	    \end{itemize}
	    \item What is the association between the IT industry and gaming industry?
	    \begin{itemize}
	        \item A 2009 study \cite{betsy_questioning_2009} found that the link between gaming and working in the IT industry was not strong, even though the gender disparity between both industries is similar.
	        % https://dl.acm.org/citation.cfm?id=1536513.1536561
        \end{itemize}
    \end{itemize}
\end{frame}

\begin{frame}
\frametitle{Any questions?}
Any questions?
\end{frame}

%\nocite{*}
\bibliographystyle{plain}
\bibliography{bibliography}

\end{document}