\documentclass{scrartcl}

\usepackage[hidelinks]{hyperref}
\usepackage[none]{hyphenat}
\usepackage{setspace}
%\doublespace TODO switch back when done

\usepackage{graphicx}
\usepackage{float}
\graphicspath{{images/}}

\newcommand{\source}[1]{\caption*{Source: {#1}} }
% Above code sourced from Xavi, Stack Overflow, https://tex.stackexchange.com/questions/95029/add-source-to-figure-caption @ 20/03/2018



\title{What are the most effective strategies to improve gender balance in game studios?}
\subtitle{COMP230 - Ethics}
\date{\today}
\author{1707981}

\begin{document}
\maketitle
\pagenumbering{arabic}

%In competitive games such as ???? -- such as what?

%There are lots of games out there with female protaganists/playable charachters with good writing

%A lot of X% -- need to add in those numbers!

%No section for a conlusion? -- Or is that the Limitations section??

%nothing so far for Benefits of diversity section -- I feel that this would be a very important section for this essay

%Overall well written so far, of course needs more completing in quite a few areas.

%mention on wether you are looking at specific role(s) within the industry or all roles in general
%as there will be a higher number of female artists in the industry compared to female programmers for example

\begin{abstract}
The gender balance in studios is predominantly male at more than 80\% of the workforce in the UK, matched similarly by the IT industry as a whole. 
\end{abstract}

\section{Introduction}
% **(Paragraph about games being for everyone but made by men)**
Games are played almost equally by all genders, with stats suggesting that women, at the least, comprise 45\% of the video-gaming audience in the U.S \cite{duggan_gaming_2015, noauthor_essential_2018}. However, in the game development industry, men comprise around 80\% of the workforce \cite{igda_satisfaction_2017}, a statistically significant difference between producer and consumer.

% **The intent of this essay**
This essay will explore potential solutions to the lack of women in the games industry, and how such solutions could be leveraged and combined to facilitate more diverse work environments.

\section{Background}
\subsection{State of the industry}
% About the proportion of women in the games industry
According to IGDA, women comprised 21\% of the games industry workforce in 2018 \cite{igda_satisfaction_2017}, down 1\% from 22\% in 2014 \cite{igda_satisfaction_2014}. This figure drops to x in the UK. Similarly, in 2018, women comprised only 17\% of IT sectors in the UK \cite{uk_employees_2018}. Furthermore, according to HESA, only 17\% of computer science university students were female in 2016 \cite{hesa_2016}, seemingly corresponding exactly with the industry. In combination of the two related minorities, female game programmers fared worse, with only 11\% \cite{igda_diversity_2016} of women in the industry, or 3\% of the total workforce, taking programming roles.

https://computinged.wordpress.com/2014/10/30/npr-when-women-stopped-coding-in-1980s-are-we-about-to-repeat-the-past/

\subsection{A question of interest}
% Why this is the case
The lack of female developers is sometimes attributed to a proportional lack of interest in the role. While this is difficult to quantify, the past suggests it is unlikely: In 1985, women peaked at 35\% of the IT workforce in the U.S \cite{vogel_spitting_2017}. A decline has followed, with explanations ranging from the emergence of the white male nerd stereotype \cite{vogel_spitting_2017} to the extreme working hours of industry \cite{allan_fair_2017}.

%It is difficult to quantify interest, and which specific factors determine it. This essay assumes that there is no moral obligation to generate greater interest of games in the female audience. However, potential social or cultural sources that devalue or demotivate this audience, such as female-oriented harassment \cite{allan_fair_2017, gonzalez_entering_2014}, homogeneous cultures, or sexism, can be presented as potential causes of decreased interest.

\section{Is it a problem?}
(INSERT GAME GENRE FIGURES HERE WOULDN'T IT BE NICE IF THIS WERE REASONABLY EASY TO DO IN LATEX)

% Reduced market appeal
The lower proportion of women in industry could predict a reduced investment in appealing to the female audience. According to to Statista, up to 52\% of game developers are working on Action games \cite{noauthor_game_nodate}, whilst according to Quantic Foundry, only 20\% of Action RPG players are female (18\% for Action-Adventure) \cite{yee_beyond_2017}. The market itself is also driven largely by the shooting and action genres, taking 25\% and 22\% of sales respectively in the U.S. \cite{noauthor_u.s._nodate} (although this does not account for revenue made through microtransactions). These figures come back around in the production costs of games, wherein the top 10 most expensive games were some form of shooter, action game, or RPG in 2016 \cite{statista_expensive_2016}.

% Reduced market investment
It is therefore inferred that the majority of high-investment, high-selling and (theoretically) high-quality games are targeted towards males despite the well-balanced playerbase. In a capitalist society, an expected consequence of this is the continued targeting towards males, as they comprise the biggest source of revenue in retail. Games would, as such, be designed to appeal to male players.

% Differences between preferences
Yet research suggests that male and female players perceive games in different ways, reflecting preferences for different game attributes \cite{yee_beyond_2017} including identification with character's gender and personality qualities \cite{7782511}, in a market where female characters have been described as `of secondary nature' \cite{gonzalez_entering_2014}. It is arguable that women are most statistically likely to share the attributes which differ from men \cite{yee_beyond_2017}. This is the basis for an argument that the appeal of gaming for women and girls could be improved by increasing their participation in industry.

\section{Exploring Solutions}
\subsection{Hiring quotas}
% Proposing hiring quotas
Hiring on the basis of gender, if for a greater purpose, is discrimination by U.K. definition \cite{discrimination}. However, it is questioned whether discrimination is the cause of the imbalance to begin with. If hiring bias against women exists, taking legal measures to remove this bias would be an arguably essential solution. However, the existence of a hiring bias for or against women is difficult to prove, and existing views are mixed, with some female employees believing that a company is in fact more likely to hire a woman with similar skills, as they wish to improve the gender balance \cite{allan_fair_2017}.

Meanwhile in Australia, after conducting a trial, a government study cautioned against anonymising worker applications, as women were actually 2.9\% \textit{more} likely to be hired when it was clear they were female \cite{noauthor_going_nodate}, indicating a bias in the opposite direction. However, this figure does not isolate game companies - unique for their male-dominated culture - from other companies, and further research could be done to identify the effects of blind recruitment in game studios to identify the prevalence of discrimination.

% Ethics of hiring based on gender
However, both arguments may be irrelevant: higher education is almost equally inbalanced, since all 88\% of game industry employees studied a degree relevant to game development \cite{igda_satisfaction_2017}, this is the most likely area where intervention should begin.

\subsection{Marketing to women}
Depictions of an educational course have a subconscious influence on potential applicants. A 2018 experiment \cite{Metaxa-Kakavouli:2018:GDS:3173574.3174188} tested the effects of two versions of a web page for a computer science course. Half the sample were presented a gender-neutral background of leaves in a tree, while the other showed the cast of \textit{Star Trek} in an contemplating pose. The latter resulted in a negative response, notably a lower sense of belonging, and ultimately a reduction of female applications to the hypothetical course by 15\%.

This ``sense of belonging'' is an important motivator in human psychology, believed to enhance meaning in life \cite{lambert_belong_2013}. The experiment suggests that the perception of the industry is an influential factor. This has a historical basis. An insight into the tech industry (\cite{vogel_spitting_2017}) discusses computing in 1985, when women comprised 35\% of the workforce in the U.S. The increase in the proportion of women was associated with a push for diversity, backed by overt marketing campaigns depicting more equal numbers of men and women in the workplace.

Changing the reality of the industry could begin by changing the perception of it. A relevant but dated study on a WCI (Women and Computing Initiative) at a Norwegian University \cite{doi:10.1177/0306312706063788} extensively documented this process in the early 2000's. While students did not particularly favour the depiction of computing as ``particularly fitting for women'' (with one interviewee, Kaja, complaining of `all the focus on women'), they felt that the atmosphere - with up to 35\% of female students - felt more positive. Many, however, did not favour the women-only computer lab, which they felt was unfair to male students, some of whom they collaborated with.

It is understandable that one would find the exclusion of male students unfair - especially when in the minority group, which is arguably, based on their experience, most likely to be empathetic of this. Deliberately gender-neutral marketing could be an effective tool for improving the perspective of the games industry.  'women developers' and just want to be 'developers'.

\subsection{Combating harassment}
% The harassment problem and its causes
The regularity of workplace harassment \cite{allan_fair_2017, gonzalez_entering_2014}, such as those raised in the recent lawsuit of Riot Games \cite{saba_negron_nodate} - a highly reputed online game company - suggests that of harassment women is not adequately addressed in industry. This extends into the playerbase as well, and well-documented cases of regular incidents of harassment of women by vocal portions of this gaming community, such as \#GamerGate, \cite{Chatzakou:2017:HBS:3078714.3078721, gonzalez_entering_2014, allan_fair_2017}, have publicly illustrated the frequency and weight of harassment of women in gaming.

Even if the homogeneity of the games industry could be proven not a problem, the reputation of the industry as one particularly aggressive (or even passive-aggressive (cite)) against women should be treated as a heavy ethical concern and combated. In fact, rather than being a solution, heightened awareness could be a further \textit{cause} of the imbalance, as it presents the image of the industry as a threatening and dangerous one.

\section{Conclusion}


\bibliography{bibliography} 
\bibliographystyle{ieeetr}

\end{document}

========== 15% Depth of Insight into Ethics and Professionalism ==========
Extensive depth of insight into ethical conduct and professional values.

Critical insights explore, in detail, the pertinence of a specific and important ethical and/or professional challenge in digital games development.

Insights highlight a potentially effective solution to address the challenge

========== 15% Adequacy of Analysis of Research Articles ==========
Extensive analysis has been presented.

The limitations of most analyses is explicitly acknowledged.

Some limitations are addressed.

========== 10% Relevance of and Focus on the Research Question ===========
Extensive focus on a single, specific research question.

The research question is explicitly defined. 

Conclusion explicitly refers back to the question. 

Appropriate references to case studies and examples of issues in the games industry are used to argue and demonstrate the relevance of the research question.

========== 5% Appropriateness of Academic Writing ==========
Significant evidence of mastery of academic writing.

The reference section is complete and well-formed in either ACM or IEEE format.

All in-text citations and quotations are correct.

========== 5% Specificity, Verifiability & Accuracy of Claims ===========
All claims have a clear source of evidence.

Almost no errors and/or misinterpretations.

Appropriate forms of evidence are used to support many claims

========== 5% Appropriateness of Spelling & Grammar ==========
No spelling or grammar
errors.

Active voice is prevalent.

Grammar is leveraged deliberately to draw attention to salient points.

========== 5% Appropriateness of Essay Structure =========
There is considerable structure, leveraged to effectively highlight the argument and key takeaway points.

All sentences and paragraphs are well constructed.

There is a clear and well-constructed introduction and conclusion