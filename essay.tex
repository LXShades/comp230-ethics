\documentclass{scrartcl}

\usepackage[hidelinks]{hyperref}
\usepackage[none]{hyphenat}
\usepackage{setspace}
%\doublespace TODO switch back when done

\usepackage{graphicx}
\usepackage{float}
\graphicspath{{images/}}

\newcommand{\source}[1]{\caption*{Source: {#1}} }
% Above code sourced from Xavi, Stack Overflow, https://tex.stackexchange.com/questions/95029/add-source-to-figure-caption @ 20/03/2018

\title{What affects the low participation of women in the game development industry and how could this change?}
\subtitle{COMP230 - Ethics}
\date{\today}
\author{1707981}

\begin{document}
\maketitle
\pagenumbering{arabic}

\begin{abstract}
There are no conclusions yet lol. There might never be. Genders have been fighting for generations as have most different people have with other different people.
\end{abstract}

\section{Introduction}
The low participation of women in the game industry is both a trending topic and a dated one. For many years, the game industry has been predominantly driven by men. 

In the past, the voices of women and men in the media were arguably moderated by the higher-ups - which happen to be primarily men. However, in the past decade and a half, the uprising of social media has created an outstanding platform for women, minorities, and less powerful people to speak with less censorship and risk of physical oppression. (citations will be challenging here) On these platforms, recent movements such as 'GamerGate' and 'MeToo' have raised large-scale discussions and accounts of oppression in the media industry.

As an ethical topic, an exhaustive analysis of all information, biases, and perspectives is impossible. For example, when comparing demographics and attitudes in different workplaces it will be impossible to address every single bias.

\section{}

\section{Gender ratio: Influence on Games and Audiences}
An anecdotal hypothesis claims that the proportion of female players could be linked to the proportion of female developers. However, the confidentiality of company demographics makes it difficult to verify this.

Look into the psychology of it.

\section{Stuff learned from papers}

\section{Potential conclusions}
Programmes introduced to girls at an early age have been successful in driving interest and potential self-identification of an aptitude in computing abilities.

\section{To write about}
- A graph weighing the participation of women in game-oriented university courses against studios
 - Try to squeeze out dates if possible to make it a running graph
 - Look at Asian games' popularity in the West compared to Western games' popularity in the East, and consider arguing for a similar link between men and women.
 - Games are dominated by men - but only 'hardcore' games. In fact, the majority of games are dominated by women.
 - Is this a crisis or natural selection?
 - Is it biological, or is it culture? How are the two linked?

Condensed:
- The hiring
- The socials
 - The cultures
 - The harassments
- The biologicals
- The demand

\bibliography{references} 
\bibliographystyle{ieeetr}

\end{document}